\documentclass[12pt]{article}

\usepackage{amsmath}
\usepackage{amsfonts}
\usepackage{amssymb}
\usepackage{hyperref}
%\usepackage{a4wide}
\usepackage{epsfig}
\usepackage[a4paper, total={7.5in, 10in}]{geometry}
\begin{document}
\begin{minipage}{.3\textwidth}%
\end{minipage} \hfill
\raisebox{1.5mm}{%
\begin{minipage}{0.69\textwidth}\sf\flushright%
\textbf{\Large Introduction to cryptography}\mbox{\hspace{5mm}}\\ {ESISAR IR\&C 2020/2021}\mbox{\hspace{5mm}}%
\end{minipage}%
}\\\hrule \ \\
\begin{minipage}[t]{0.5\textwidth}\sf\textit{Hugounenq Cyril}
\end{minipage} \hfill%
%  \vspace{-0.5em}
%  \begin{center}
%	\textbf{\Large } \\[0.5ex]
%\end{center}
\vspace{0.5em}

All useful documents on Pari-GP can be found at  {\bf \url{https://pari.math.u-bordeaux.fr/doc.html}}

\section{Getting familiar with Pari-GP: Generating primes}
\begin{enumerate}
	\item Start Pari-GP with the command {\it \bf gp} and import file {\it rsa.gp} by entering {\it read(rsa)} or {\it \textbackslash r rsa.gp}
	\item Use {\it gen\_prime} to generate a prime number on 512 bits. Using the User's guide, explain why the generation is so fast.
	\item Is it secure to use such a function to generate a long term RSA key? Modify the existing code if necessary.
	\item In practice, NIST recommends stronger assumptions on primes used for RSA
	\begin{itemize}
		\item {\bf Strong prime:} $p$ is a strong prime iff $\frac{p-1}{2}$ is also prime
		\item {\bf RSA prime:} $p$ is a RSA prime iff $\frac{p-1}{2}$ is a strong prime
		\item Implement a function {\it gen\_RSA\_prime(b)} which generates a RSA prime on exactly $b$ bits
	\end{itemize}
	\item {\it Bonus:} Implement a function counting the number of primes lower than a bound $m$ and check empirically that this number is close to $\frac{m}{log(m)}$
\end{enumerate}

\section{Generating RSA parameters}
\begin{enumerate}
	\item Implement a function {\it gen\_RSA\_parameters(b)} which generates a set of RSA parameters on $b$-bits and returns $[N,e,d]$.
	\item Implement a function {\it RSA\_encrypt(M,N,e)} which encrypts a message $M$ using a public key $[N,e]$
	\item Implement a function {\it RSA\_decrypt(C,N,d)} which decrypts a message $C$ using a private key $[N,d]$
	\item {\it Bonus:} Using the gp function {\it chinese}, implement a function {\it RSA\_sign\_CRT(M,N,$d_p$,$d_q$,p,q)} which signs a message $M$ using a private key $[N,d \mod p-1, d \mod q-1,p,q]$ (modify the function {\it RSA\_gen\_parameters} accordingly). The signature value shall be $S=M^d [N]$.
\end{enumerate}

\section{Attacks on RSA}
\begin{enumerate}
	\item Implement a function {\it factor\_from\_phi(N,phi)} which factors $N$ given $\varphi(N)$.
	\item Implement a function {\it factor\_from\_d(N,e,d)} which factors $N$ given $e$ and $d$.
	\item Implement a function {\it broacast\_attack($M^3 \mod N_a,N_a,M^3 \mod N_b,N_b,M^3 \mod N_c,N_c$)} which recovers $M$.
	\item Implement a function {\it wiener\_attack(N,e)} which recovers $d$ assuming that $d < \frac{1}{3}N^{\frac{1}{4}}$.
\end{enumerate}

\end{document}
