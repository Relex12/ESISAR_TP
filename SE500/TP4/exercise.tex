\documentclass[12pt]{article}

\usepackage{amsmath}
\usepackage{amsfonts}
\usepackage{amssymb}
%\usepackage{a4wide}
\usepackage{epsfig}
\usepackage[a4paper, total={7.5in, 10in}]{geometry}
\begin{document}
\begin{minipage}{.3\textwidth}%
\end{minipage} \hfill
\raisebox{1.5mm}{%
\begin{minipage}{0.69\textwidth}\sf\flushright%
\textbf{\Large Introduction to cryptography}\mbox{\hspace{5mm}}\\ {ESISAR IR\&C 2019/2020}\mbox{\hspace{5mm}}%
\end{minipage}%
}\\\hrule \ \\
\begin{minipage}[t]{0.5\textwidth}\sf\textit{Cyril Hugounenq}
\end{minipage} \hfill%
%  \vspace{-0.5em}
%  \begin{center}
%	\textbf{\Large } \\[0.5ex]
%\end{center}
\vspace{0.5em}

\section{Playing with ECC}
\begin{enumerate}
	\item Init a prime number $p=randomprime(2^{100})$ and the elliptic curve $E:y^2=x^3+2x+3$ over $\mathbb{Z}/p\mathbb{Z}$ using the function $ellinit$
	\item Draw a random point on the curve $P=random(E)$ and check that $P$ is indeed on the curve using two methods
	\item Draw a second point $Q$ on the curve and check that $R=P+Q$ is also on the curve
	\item Using the functions $ellcard$ and $ellorder$, check that the order of $P$ divides the cardinal of $E(F_p)$
	\item Using the function $ellmul$, check that $ellorder$ returns the order of a point.
	\item Can $E(F_p)$ admit points of 2-torsion? Of 3-torsion?
	\item Use the function $ellgenerators(E)$ to generate a point $G$. What is the order of $G$?
	\item {\it Bonus:} For a fixed value of $p$ prime and using $ellcard$, check empirically Hasse theorem on multiple curves. What is the empirical distribution of cardinals?
\end{enumerate}

\section{ECDSA}
\begin{enumerate}
	\item Using the notations given in the course, implement the functions $ECDSA-SHA256-sign(E,G,n,d_a,M)$ and $ECDSA-SHA256-verify(E,G,n,Q_a,M,s,x_r)$
\end{enumerate}

\section{Rho-Pollard on ECC}
\begin{enumerate}
	\item Let $E$ be an elliptic curve, $G$ a point on $E$, and $P=aG$ for some unknown $a$. Write the function $rho-pollard-ECC(E,G,P)$ implementing the following algorithm
	\begin{itemize}
		\item For a point $Q=(x,y)$ on $E$, we define
		\[ f(Q) =
			\left\{
				\begin{array}{lll}
					Q + P&if & x \text{ mod } 3 = 0 \\
					2Q   &if & x \text{ mod } 3 = 1 \\
					Q + G&if & x \text{ mod } 3 = 2 \\
				\end{array}
			\right.
		\]
		\item We define the sequences
		\[
			\begin{array}{ll}
				R_0=P,& R_{i+1}=f(R_i) \\
				S_0=P,& S_{i+1}=f(f(S_i)) \\
			\end{array}
		\]
		\item If $R_i = a_iP + b_iG$, what is the value of $a_{i+1}$ and $b_{i+1}$?
		\item Compute the sequences $R_i$ and $S_i$ until $R_i=S_i$.
		\item Return $a$.
	\end{itemize}
	\item What is the empirical complexity of this attack with respect to the order of $G$?
\end{enumerate}

\end{document}
